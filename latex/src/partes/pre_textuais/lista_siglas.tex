\begin{siglas}
    \item[IPFS] InterPlanetary File System - Sistema de Arquivos Interplanetário
    \item[Freelancer] É um termo para descrever o profissional autónomo que realiza
serviços para outras empresas.
    \item[Workana] É uma empresa que fornece uma plataforma de Freelance.
    \item[Upwork] Assim como o Workana, é empresa que fornece uma plataforma de Freelance.
    \item[Hash] É o resultado de, ou função que, gera um mesmo resultado a partir de uma entrada de dados.
    \item[FTP] É um protocolo de transferência de arquivos entre servidor e cliente.
    \item[SMTP] É um protocolo de transmissão de e-mails.
    \item[HTTP] É um protocolo de de transferência de hiper-texto, como o HTML.
    \item[Framework] É um conjunto de ferramentas e códigos que um desenvolvedor pode utilizar para facilitar o desenvolvimento de um projeto. Os exemplos são Angular, Vue, .NET, etc.
    \item[Javascript] É a implementação da especificação ECMASCRIPT, no qual é uma linguagem não tipada e fácil de ser usada que é interpretada por todos os navegadores.
    \item[Typescript] É um super-conjunto estritamente tipado do Javascript, no qual adiciona novas funcionalidades e principalmente a tipagem a linguagem do Javascript.
    \item[Angular] É um Framework escrito em Typescript desenvolvido pela Google para desenvolvedor aplicações web.
    \item[HTML] É a sigla para HyperText Markup Language, é uma linguagem usada para definir a estrutura de uma página web.
    \item[CSS] É a sigla para Cascading Style Sheets, que é um mecanismo de aplicar estilos visuais para uma página HTML.
    \item[SCSS] É um pré-processador de CSS, que adiciona novas funcionalidades ao CSS de forma a facilitar o desenvolvimento.
\end{siglas}